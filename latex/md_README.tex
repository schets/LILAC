L\-I\-L\-A\-C\-: Learning and Integration of Lasers for Adaptive Control

This project provides a framework with which to analyze/control lasers, and more generally, any tunable dynamical system.

You can find documentation on the \href{http://github.com/schets/LILAC/wiki}{\tt wiki}, and doxygen generated \href{http://schets.github.io/LILAC}{\tt advanced documentation} as well

Installation\-: This project depends on \href{gcc.gnu.org}{\tt G\-C\-C}, \href{git-scm.com}{\tt Git}, \href{www.fftw.org}{\tt F\-F\-T\-W}, \href{http://www.gnu.org/software/gsl/}{\tt G\-S\-L} and \href{eigen.tuxfamily.org}{\tt Eigen}. These are simple to install on Linux and Macintosh systems and each site provides installation instructions. G\-S\-L will likely be dropped as a requirement in the future, but for now it useful functions and allows more time to focus on the actual engine

Once you have installed G\-C\-C, Git, F\-F\-T\-W, G\-S\-L, and Eigen, you can proceed to download and compile L\-I\-L\-A\-C. L\-I\-L\-A\-C can be downloaded without git by going to the github page and manually downloading the file, but using git as a version control is much more convinient and less error prone.

On Linux\-:


\begin{DoxyEnumerate}
\item From the command line, proceed to the directory in which you want to have lilac installed.
\item Run the command\-: git clone \href{https://github.com/schets/LILAC}{\tt https\-://github.\-com/schets/\-L\-I\-L\-A\-C} lilac
\item Enter the lilac directory (cd lilac)
\item Compile the code by running make
\item Upon successful compilation, the lilac binary can be found in ithe directory bin
\end{DoxyEnumerate}

See the wiki for a tutorial on writing input files, and extending the engine in various manners. The engine is designed so that only a cursory knowledge of C/\-C++ is needed to write most extensions, and in general extending the code will not involve editing the engine itself.

A nice tutorial on C++ can be found at www.\-cplusplus.\-com/tutorial, and a tutorial on pointers (arrays in C/\-C++) can be found at \href{http://pw1.netcom.com/~tjensen/ptr/pointers.htm}{\tt http\-://pw1.\-netcom.\-com/$\sim$tjensen/ptr/pointers.\-htm} 